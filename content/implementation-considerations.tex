\chapter{Consideraciones sobre la implementación}

\section{Visión general}

Este capítulo detalla los aspectos más importantes a tener en cuenta para codificar el software.

\section{Entorno de desarrollo}

Estas son las herramientas utilizadas para la realización de este proyecto.

\subsection{Visual Studio}

Microsoft Visual Studio\cite{VisualStudio} es un entorno de desarrollo integrado creado por Microsoft, se usa para desarrollar programas para Microsoft Windows , así como páginas, aplicaciones y servicios web. Visual Studio usa plataformas de desarrollo de software como Windows API, Windows Forms, Windows Presentation Foundation, Windows Store y Microsoft Silverlight. Puede producir tanto código nativo como gestionado.

\begin{figure}[!htp]
	 \centering
	 \includegraphics[scale=1.0]{fig/visualstudio_logo}
	 \caption{Logotipo de Visual Studio}
\end{figure}

Visual Studio incluye un editor de código que soporta IntelliSense y refactorización de código. El debugger integrado funciona como un debugger a nivel de código fuente y a nivel de código máquina. Otras herramientas integradas que incluye Visual Studio son diseñadores para web, clases y esquemas de \acrshort{bd}. Acepta plug-ins que extienden la funcionalidad a casi cualquier nivel, incluido soporte añadido para sistemas de control de fuente como Subversion o la adición de sets de herramientas como editores y diseñadores visuales para lenguajes específicos de dominio o para otros aspectos del ciclo de desarrollo de software.

Visual Studio soporta varios lenguajes de programación y permite al editor y debugger soportar casi cualquier lenguaje de programación, siempre y cuando exista un servicio específico de ese lenguaje, los lenguajes integrados por defecto en Visual Studio incluyen C, C++ y \acrshort{cppcli}, VB.NET, C\# y F\#. Soporte para otros lenguajes como Python, Ruby, Node.js y M entre otros está disponible mediante servicios de lenguaje a instalar por separado, también soporta \acrshort{xml}/\acrshort{xslt}, \acrshort{html}/\acrshort{xhtml}, \acrfull{js} y \acrshort{css}. Java (y J\#) fueron soportados en el pasado.


\subsection{Google Chrome}

Google Chrome\cite{Chrome} es un explorador web gratuito desarrollado por Google. Usaba el WebKit layout engine hasta la versión 27 y con la excepción de sus versiones de iOS, desde la versión 28, Chrome usa Blink, fue en un principio lanzado como una versión beta para Microsoft Windows el 2 de septiembre de 2008 y finalmente como una versión estable el 11 de diciembre de 2008.

\begin{figure}[!htp]
	 \centering
	 \includegraphics[scale=0.3]{fig/googleChrome_logo}
	 \caption{Logotipo de Google Chrome}
\end{figure}

A fecha de marzo de 2016, StatCounter estima que Google Chrome tiene un 60.1\% del mercado de los navegadores web como navegador de sobremesa, es también el navegador más usado en smartphones, su éxito ha dado lugar a que Google expanda la marca “Chrome” a otros productos como el Chromecast o Chromebook.

\subsection{Git}

Git\cite{Git} es un sistema de control de versión y es ampliamente utilizado en el ámbito del desarrollo de software y otras tareas de control de versión. Es un sistema distribuido con el énfasis puesto en su velocidad, integridad de datos y el soporte para flujos de trabajo distribuidos y no lineales. Git fue creado por Linus Torvalds en 2005 para el desarrollo del kernel Linux, con la colaboración de otros desarrolladores del kernel para su desarrollo inicial.

\begin{figure}[!htp]
	 \centering
	 \includegraphics[scale=0.2]{fig/git_logo}
	 \caption{Logotipo de Git}
\end{figure}

Como con otros sistemas de control de versión distribuidos, y a diferencia de en la mayoría de sistemas cliente-servidor, cada directorio de trabajo Git es un repositorio con historial y capacidades de seguimiento de versión completos, independientemente del acceso a red o de un servidor central, como el kernel Linux, git es gratuito y es distribuido bajo los términos de la GNU General Public License versión 2.