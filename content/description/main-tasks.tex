\subsection{Tareas principales}

La implantación del proyecto comprende las siguientes tareas o actividades: 

\begin{itemize}

	\item \textbf{T1}	

	Definición del problema: En esta tarea se define el problema y se establecen una serie de ideas fundamentales para el proyecto, por ejemplo, el número de subsistemas por el que va a estar compuesta la solución. 
	\item \textbf{T2}	

	Especificación de requisitos: En esta tarea se deciden los requisitos funcionales y no funcionales que debe cumplir cada pieza de software a desarrollar en el proyecto.

	\item \textbf{T3}	

	Especificación funcional: En esta tarea se hace un diseño de alto nivel del software a implementar.

	\item \textbf{T4}

	Diseño del software: En esta tarea se detalla cómo se va a trabajar sobre las plantillas de software existente para crear un producto que satisfaga las especificaciones.

	\item \textbf{T4.1}

	Diseño del lado cliente.

	\item \textbf{T4.2}	

	Diseño del lado servidor.

	\item \textbf{T5}

	Implementación: En esta tarea se codifica todo lo diseñado en la tarea 4.
	
	\item \textbf{T5.1}
	
	Implementación de la funcionalidad base del cliente.
	
	\item \textbf{T5.2}
	
	Implementación de la funcionalidad base del servidor.
	
	\item \textbf{T6}
	
	Integración: En esta tarea se codifica todo lo necesario para que los múltiples sistemas incluidos en la solución trabajen juntos.
	
	\item \textbf{T6.1}
	
	Integrar cliente con servidor y \acrshort{api} de OpenWeatherMap.
	
	\item \textbf{T6.2}	
	
	Integrar servidor con Microsoft Cognitive Services y Google \acrshort{api}s.
	
	\item \textbf{T7}	
	
	Adquisición de material para el Smart Mirror.
	
	\item \textbf{T7.1}	
	
	Adquisición de marco para Smart Mirror.
	
	\item \textbf{T7.2}
	
	Adquisición de espejo de observación.
	
	\item \textbf{T7.3}	
	
	Montaje con monitor.
	
	\item \textbf{T8}	
	
	Validar y verificar software.
	
	\item \textbf{T9}	
	
	Creación de guía de usuario.

\end{itemize}