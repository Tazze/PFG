\chapter*{Resumen}

El proyecto consistirá de una Raspberry Pi 2 Model B con Windows 10 IoT Core y un servidor en la plataforma Azure de Microsoft, la Raspberry mostrará datos generales como el tiempo o datos dependientes del usuario tales como sus emails sin leer, sus tareas y sus eventos de calendario y determinará el usuario haciendo uso de una WebCam y Project Oxford para reconocimiento facial.

Los usuarios dispondrán de una interfaz Web para entrar en su cuenta Google y autorizar al Smart Mirror para acceder a sus datos, asimismo, el servidor que hospeda la interfaz web se encargará de recoger la información de las diferentes API (Google Mail, Google Tasks, Google Calendar, el tiempo, Project Oxford) y suministrársela al espejo cliente.

Debido al planteamiento de la solución se pueden añadir más smart mirror sin que ningún usuario tenga que volver a configurar nada, y serán reconocidos por cualquier smart mirror siempre y cuando todos usen el mismo servidor como base.
La Raspberry tendrá, aparte de la pantalla y la WebCam un circuito electrónico con un interruptor cinético sensible que se activará al dar un toque al marco del smart mirror, todos los componentes estarán encapsulados en un marco con un espejo de dos caras que dejará pasar la luz de la pantalla al otro lado, mostrando la información al usuario en la superficie reflectante.


\vspace{2em}

{\Large\bfseries\sffamily Descriptores}
\vspace{3\medskipamount}

Smart Mirror, ASP.NET, \acrshort{iot}, \acrshort{uwp}.