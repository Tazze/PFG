\chapter{Introducción}\label{cha:introduccion}

\section{Presentación del documento}

El presente informe describe el proyecto de desarrollo \emph{Diseño e implementación de un Smart Mirror multiusuario con autenticación por reconocimiento facial}, cuyo objetivo es crear un Smart Mirror en el que cualquier usuario pueda ver información sobre su perfil online de forma rápida y sencilla, también se incluye una definición de objetivos, fases, actividades y recursos del proyecto.

El contenido de este documento se estructura en torno a los siguientes puntos.

\begin{itemize}
	\item \textbf{Objetivos del proyecto:}

	Definición general del proyecto y detalle de sus objetivos.

	\item \textbf{Especificación de requisitos:}
		
	Idea general del software a desarrollar asi como informacion sobre su funcionalidad y sus requisitos tanto funcionales como no funcionales.
	
	\item \textbf{Tecnologías utilizadas:}
		
	Información sobre todas las tecnologías que el software implementado va a usar.
	
	\item \textbf{Especificación del diseño:}

	Definición del código a implementar en la solución software.

	\item \textbf{Consideraciones sobre la implementación:}
	
	Descripción de aspectos importantes para la codificación del software.

	\item \textbf{Plan de pruebas:}

	Definición el proceso de verificación del software.

	\item \textbf{Manual de usuario:}
	
	Guia para el uso correcto del software.

	\item \textbf{Incidencias:}
	
	Problemas surgidos durante el desarrollo del proyecto y su solución

\end{itemize}

\section{Motivación}

Este proyecto nace de la necesidad de un dispositivo que pueda mantener a un usuario conectado con su vida online, sin resultar una distracción para el mismo, por tanto, se procedió al diseño de una solución que fuera capaz de mostrar información relevante para el usuario sin necesidad de interacción excesiva por su parte, siendo solo necesaria una configuración inicial.

\section{El concepto de Smart Mirror}

Un Smart Mirror suele estar compuesto fundamentalmente de un monitor situado detrás de un espejo de observación de dos caras, de tal manera que la luz del monitor sea lo único que se vea sobre la superficie reflectante, creando una ilusión de que el contenido del monitor está plasmado en la superficie de un espejo normal y corriente, la función del Smart Mirror es proporcionar información útil para el usuario y en algunos casos, interacción sin necesidad de dispositivos adicionales como un Smartphone o un ordenador, algunos ejemplos de Smart Mirror incluyen el Connected Mirror que presentó BMW en \acrshort{ces} 2016, u OpenIspilu, el proyecto ganador del Concurso de Aplicaciones de Open Data Euskadi 2016, en el caso de este proyecto, la función del Smart Mirror es proporcionar la información necesaria a un usuario para poder prescindir al máximo de su teléfono móvil.