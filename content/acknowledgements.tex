\chapter*{Agradecimientos}
\addcontentsline{toc}{chapter}{Agradecimientos}

Este es un apartado para hacer mención a todas aquellas personas sin las cuales no estaría escribiendo estas palabras:

\begin{itemize}

\item \textbf{Mis Padres:} por primera vez desde que empecé a escribir esta memoria, me encuentro sin palabras para describir lo que quiero expresar, mi padre me ha proporcionado una educación con la que muchas personas solo pueden soñar como un simple empleado de una caja de ahorros, y aunque el apoyo moral de ambos no ha sido siempre el más adecuado en mi opinión, prefiero quedarme con la certeza de que hasta sus peores errores han nacido de su amor incondicional por mí, y por eso siento que tengo una deuda para con ellos que nunca podré saldar.

\item \textbf{Mis buenos compañeros de titulación:}
\begin{itemize}
\item \textbf{Jesus María Sesma Solance:} sus conocimientos me han salvado más de una vez, y esta no es una excepción, desde aquel proyecto de desarrollo avanzado de software en el que se podría decir que lo único que hice fue teclear el código que él me decía que escribiese hasta esta misma memoria, codificada en LaTeX tomando como ejemplo la suya propia que me prestó sin tener que pedírselo dos veces.
\item \textbf{Sergio Ausín Ortega:}  le debo bastante, ofreciendo una mano siempre que me he encontrado con un escollo aparentemente insalvable, por ejemplo, aquella vez que consiguió hacer funcionar la base de datos en nuestro proyecto de Software Process and Quality, aspecto del que yo estaba encargado, o todas esas veces que he podido recurrir a él cuándo he sido incapaz de comprender algún aspecto de ASP.NET
\item \textbf{Todos los demás:} Verónica Merino, Unai Alonso, Álvaro García, Jon Ander Novella, Jesús Pereira, Irene Díez, Alejandro Pérez Carballo… la lista no acabaría nunca, a ellos les agradezco el gran ambiente del que he disfrutado durante estos años y que ha servido como contrapunto relajado a la exigencia del mundo académico.
\end{itemize}

\item \textbf{Mi tutor de proyecto, Diego López de Ipiña:} he de decir que a la hora de presentarle la idea del proyecto dudaba de su potencial, pero el entusiasmo que mostró por el concepto me convenció para llevarlo a cabo, a él le debo que haya realizado este proyecto y todo lo que he aprendido durante su desarrollo.
\item \textbf{Todos mis profesores:} para exasperación de muchos de ellos, lo siento, pero lo más probable es que recuerde muy poco de la materia que estudie en su asignatura, pero siempre me he llevado algo de las clases a las que he atendido, y estoy convencido de que sin esas pequeñas piezas no habría podido superar mi periodo de prácticas ni haber llevado a cabo este proyecto, en mi conclusión dije que había aprendido más durante el transcurso de este año que durante la carrera, y es verdad, pero también expresé que no lo habría hecho sin las bases que ellos me han aportado, y creo que mucho más que mis palabras en un papel, les reconfortará saber que he podido utilizar aquello que aprendí con ellos y saber que todo ese gritar para que se callasen los que cuchicheaban en la parte de atrás, todas esas veces que se han llevado las manos a la cabeza leyendo respuestas incorrectas que he podido dar en los exámenes y todas esas veces en las que su voz ha corrido peligro por nuestra falta de atención, todo ello, ha marcado la diferencia, al menos para mi.

\end{itemize}