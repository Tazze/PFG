\chapter{Conclusión}

En un principio este proyecto prometía ser desafiante, ya que disponía de muy poca experiencia con las plataformas utilizadas, si bien he trabajado con ASP.NET durante mis prácticas, mi trabajo hacia uso de Web Forms, y no de \acrshort{mvc}, la plataforma utilizada en este proyecto, ese cambio total de enfoque dio lugar a más de un quebradero de cabeza, por otro lado, mi experiencia con aplicaciones universales \acrshort{uwp} también era limitada, habiendo hecho solo una aplicación para leer códigos \acrshort{qr} y abrir una página mediante un WebView, sin embargo, gracias a las herramientas adquiridas durante mi titulación para estar en proceso de aprendizaje constante no me ha resultado imposible superar esos obstáculos.

Me encuentro satisfecho con el trabajo realizado, y no en poca medida gracias al software desarrollado, pero he de decir que aprecio mucho más el hecho de haberme familiarizado con tecnologías, conceptos y herramientas que hace un año me habrían parecido imposibles de siquiera entender, he conseguido ver el potencial detrás del patrón \acrshort{mvc}, una idea que he menospreciado en alguna ocasión, denominándolo ''código ravioli'' y alegando que todo estaba demasiado separado como para codificar con facilidad, he perdido la aversión a codificar aplicaciones que trabajen sobre internet que tanto me ha limitado en el pasado y me ha hecho depender del trabajo de otros de mis compañeros, ahora sé que puedo enfrentarme a conceptos radicalmente nuevos para mi sin la ayuda de un experto y aun así salir fortalecido, habiendo adquirido maestría en aquello a lo que pertenecen.

A riesgo de parecer un ingrato, siento que he aprendido más durante este año de prácticas en empresa y proyecto de fin de grado que en el resto de la carrera, aunque no dudo que sin el conocimiento adquirido durante los ya más de cuatro años de carrera, no habría sido capaz de superar este último.
